\chapter{Introduction and Motivation}

\section{Introduction}

Past Situation

CoCo as Solution & Studies

CoCo under Basel III

\begin{itemize}
\item regulators have been simultaneously working on two different fronts in the aftermath of the 2008 crisis
\item first, they wanted to reduce the impact of a single bank failure on the rest of the economy
\item second, they aimed to reduce the probability of a bank failure taking place
\item this second goal can be achieved by making sure that banks are better equipped to weather a financial storm
\item banks need to be more loss absorbing
\item this can be achieved by increasing the amount of regulatory capital 
\end{itemize}

\begin{itemize}
\item in 2011, the Basel Committee specified that a debt instrument can only be part of the regulatory capital if losses can be absorbed in such a way that tax-payers' money is not going to be needed to bail out banks {14}
\item the loss absorption is triggered by an event which forces the bond to be either converted into common equity or written off
\item according to the Basel Committee, the trigger event is the earlier of
\subitem decision by the relevant authority to use public funds and inject these into the company without which this bank would become non-viable
\subitem decision by the same relevant authority that the bank is no longer viable without a write-off on its debt
\item in both cases the trigger is in the hands of a regulatory body that supervises the bank when the trigger mechanism is called a non-viability or regulatory trigger and is a standard component of any Tier 2 or Additional Tier 1 instrument issued in the Basel III framework
\item on this non-viability trigger, contingent debt has an automatic trigger mechanism through incorporation of an accounting trigger in its terms and conditions
\item when implementing Basel III into European law, CRD IV, has for example laid out a minimum level of 5.12 \% CET1.
\end{itemize}

Current Situation

\begin{table}[h]
	\centering
	\begin{tabular}{cl}
		\toprule
			Face Value in USD bn & Issuer \\
		\midrule
			3.75 & HSBC (GB) \\
			3.63 & UBS (CH) \\
			3.15 & Royal Bank of Scotland (GB) \\
			3.00 & Barclays (GB) \\
			2.70 & UBS (CH) \\
			2.50 & Credit Suisse (CH) \\
			2.50 & UBS (CH) \\
			2.45 & HSBC (GB) \\
			2.25 & ING (NL) \\
			2.06 & Banco Santander (ESP)\\
		\bottomrule
	\end{tabular}
	\caption[Largest CoCo issues in Europe]{Largest CoCo issues in Europe from 2010 to 2016 \citep{schmuddel2016}}
\end{table}

\section{Literature Overview}


\begin{table}[H]
	\setlength{\extrarowheight}{2.5pt}
	\centering
	\begin{tabular}{>{\centering\arraybackslash}p{4cm}>{\centering\arraybackslash}p{4cm}>{\centering\arraybackslash}p{4cm}}
		\toprule
			Structural Approach & Equity Derivative Approach & Credit Derivative Approach \\
		\midrule
			\citet{pennacchi2010structural} & \citet{de2011pricing} &  \citet{de2011pricing}\\
			\citet{glasserman2012contingent}  & \citet{henriques2011making} & \\
			\citet{madan2011conic}  &  &  \\
			\citet{albul2010contingent}  &  &  \\
			\citet{sundaresan2015design}  &  &  \\
			\citet{hilscher2014bank}  &  &  \\
			\citet{buergi2013pricing} &  &  \\
		\bottomrule
	\end{tabular}
	\caption[Literature overview of valuation approaches for CoCos] {Literature overview of valuation approaches for CoCos \citep{erismann2015pricing}}
\end{table}

\subsection{Structural Approaches}

\subsection{Equity Derivative Approaches}

\subsection{Credit Derivative Approaches}

\section{Motivation}

\begin{itemize}
\item 
\end{itemize}


\section{Methodology}

\begin{itemize}
\item in the wake of the new regulatory standards proposed by EU and the Basel Committee, an investigation and analysis on how to model the value of CoCos is of great interest for both investors and regulatory authorities
\item study CoCos in detail to get a deeper knowledge about the pros and cons as well as an understanding on how to price these instruments
\item we will go through three models on how to price CoCos thoroughly and also investigate the sensitivity of the chosen pricing frameworks
\item will go through the building parts of a CoCo 
\item Chapter 2 provides a comprehensive overview of CoCos and characteristic design features to introduce the qualitative framework needed to understand the behavior, the mechanisms and most importantly the risk drivers of a CoCo
\item Real-life examples further substantiate the relevance of the framework
\item we present and describe the structure of some of the CoCos issued so far 
\item using this framework, the three mentioned pricing approaches are introduced in the following chapter both in a setting where returns are allowed to be discontinuous
\item chapter four further outlines the data requirements and parametrization techniques of the models and provides first pricing examples
\item in section 4 we carefully study the three chosen pricing frameworks and the modifications to increase accuracy
\item Chapter 5 includes a dynamics and sensitivity analysis to assess the model behaviour with respect to different pricing and design parameters
\item an empirical analysis follows in chapter 6, where the models are parametrized on one Deutsche Bank CoCo and the resulting price dynamics are qualitively and quantitavely compared to reach a conclusion and outlook in chapter 7
\end{itemize}


