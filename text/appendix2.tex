\chapter{Code - Models} \label{creditderivativeapproach}
\section{Code Repository}
The source could will be available under the following link until October 31, 2016: \url{https://drive.google.com/open?id=0BwsP9JbeT8w5bzMzVXo1VjJwNlk}

\section{Credit Derivative Approach} \label{creditderivativeapproach}

The following source code is an implementation of the credit derivative approach \citep{de2011pricing} written in R.
 
\lstinputlisting[language=R]{code/CreditDerivativeApproach.R}

\section{Equity Derivative Approach} \label{equityderivativeapproach}

The following source code is an implementation of the equity derivative approach \citep{de2011pricing} written in R.
 
\lstinputlisting[language=R]{code/EquityDerivativeApproach.R}

\section{Structural Approach} \label{structuralapproach}

The following source code is an implementation of the structural approach \citep{pennacchi2010structural} with support from \citet{codestructrural} in translating the source code of \citet{pennacchi2010structural} from GAUSS to R.
 
\lstinputlisting[language=R]{code/StructuralApproach.R}

\chapter{Code - Sensitivity Analyses}

\section{Credit Derivative Approach} \label{sensicredit}

The following source code is an implementation of the sensitivity analysis of the credit derivative approach \citep{de2011pricing} written in R.
 
\lstinputlisting[language=R]{code/DiagramsCreditDerivativeApproach.R}

\section{Equity Derivative Approach} \label{sensiequity}

The following source code is an implementation of the sensitivity analysis of the equity derivative approach \citep{de2011pricing} written in R.
 
\lstinputlisting[language=R]{code/DiagramsEquityDerivativeApproach.R}

\section{Structural Approach} \label{sensistructural}

The following source code is an implementation of the sensitivity analysis of the structural approach \citep{pennacchi2010structural} written in R.
 
\lstinputlisting[language=R]{code/DiagramsStructuralApproach.R}

\chapter{Code - Case Study}

\section{Parametrization - Structural Approach} \label{empirical}
The following source code is used to calculate important model inputs. The source code is written in R.
\lstinputlisting[language=R]{code/CopyOfcalibrateStructuralApproach.R}

\subsection{\citet{cox1985theory} Model}

The following source code is an implementation of the method as described by \citet{remillard2013statistical}. The method is used to calibrate the \citet{cox1985theory} model which is used for the structural approach pursuant to \citet{pennacchi2010structural} based on historical yield curve data. The software is an adaption of the source code as provided by \citet{calibrateCIR}.

\lstinputlisting[language=R]{code/estimateCIRParameter.R}\label{estimateCIRParameter}

\subsection{\citet{merton1974pricing} Model}
The following source code is an implementation of the \citet{merton1974pricing} model which is used to estimate the asset volatility. The code is written in R. Similar techniques are applied by \citet{stackoverflow2014code}.
\lstinputlisting[language=R]{code/estimateMertonParameter.R}\label{estimateMertonParameter}

\subsection{Jump-Diffusion Process}
The following source code is used to estimate the jump parameters after estimating the jump intensity. The code is written in R.
\lstinputlisting[language=R]{code/estimateJumpParameter.R}\label{estimateJumpsParameter}

\section{Empirical Analysis - All Approaches} \label{fuckinglastcode}
The following source code is used to calculate the estimated prices for each of the three models over the observation period. The approach minimizes in this regard the RMSE. Both the RMSE and the MAE are calculated for the observation period. The source code is written in R.
\lstinputlisting[language=R]{code/empiricalAnalysis_CD.R}

