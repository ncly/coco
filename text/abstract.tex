\begin{abstract}
Financial crises have led to higher regulatory standards on the capital adequacy of banks. Banks are required to hold more capital with loss absorbance capacities on their balance sheet. In conjunction with this development, contingent convertible bonds (CoCos) have become an attractive instrument for banks to seek new capital. The defining characteristic of CoCos is the automatic conversion into common equity or the principal write-down when a predetermined trigger is met. Loss-absorbing capital is created, which instantly improves the capital structure of the distressed bank. However, the pricing of these hybrid instruments remains opaque. In this context, the thesis scrutinizes the valuation of CoCos with equity conversion mechanism. In this context, three dominant approaches are examined: the structural approach in accordance to \citet{pennacchi2010structural}, the credit derivatives approach and the equity derivatives approach both pursuant to \citet{de2011pricing}. Additionally, the application covers sensitivity analyses to further understand the dynamics of the different methodologies. Based on a case study of HSBC's perpetual subordinated contingent convertible securities (ISIN US404280AT69) the viability of the approaches is evaluated by analyzing their pricing tracking accuracy. Subsequently, comprehensive software is provided for further applications of the aforementioned pricing approaches. %\todo{add second page before and last page}
%\todo{Regulatory treatment: high and low trigger CoCos; table with examples} %\todo{last page}
\end{abstract}