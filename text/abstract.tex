\begin{abstract}
Financial crises have led to higher regulatory standards on the capital adequacy of banks. Banks are required to hold more capital with loss absorbance capacities on their balance sheet. In conjunction with this development, contingent convertible bonds (CoCos) have become an attractive instrument for banks to seek new capital. The defining characteristic of CoCos is the automatic conversion into common equity or the principal write-down when a certain ratio meets a predetermined trigger. Loss-absorbing capital is created, which instantly improves the capital structure of the distressed bank. However, the pricing of these hybrid instruments remains opaque. In this context, the thesis scrutinizes the valuation of CoCos with equity conversion mechanisms. The paper examines three dominant approaches: the structural approach in accordance to \citet{pennacchi2010structural}, the credit derivative approach and the equity derivative approach both under \citet{de2011pricing}. Additionally, the application covers sensitivity analyses to understand the dynamics of the different methodologies. Based on a case study of HSBC's Perpetual Subordinated Contingent Convertible Securities (ISIN US404280AT69) the viability of the approaches is evaluated by analyzing their price tracking accuracy. Subsequently, the comprehensive software provides a basis for further applications of the pricing approaches.
\end{abstract}