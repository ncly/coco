\begin{abstract}
The global financial crisis has led to new regulatory standards on capital adequacy. Banks are required to hold more capital with loss absorbance capacities. In conjunction with this development, contingent convertible bonds (CoCos) have become an attractive instrument to seek fresh capital. CoCos convert automatically into common equity when a predetermined trigger is reached. Loss absorbing capital is created which instantly improves the capital structure of the distressed bank. The thesis scrutinizes how contingent convertible bonds can be priced. Three approaches are examined: (1) Equity Derivatives Approach, (2) Credit Derivative Approach, which are were both developed by , and (3) the Structural Approach pursuant to .
In an empirical analysis on a Deutsche Bank CoCo the aforementioned methods are evaluated. The application includes sensitivity analysis to further understand the dynamics of the different methodologies. Software is provided for further replication.
\end{abstract}