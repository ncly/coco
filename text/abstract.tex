\begin{abstract}
Financial crises have led to higher regulatory standards on the capital adequacy of banks. They are required to hold more capital with loss absorbance capacities on their balance sheet. In conjunction with this development, contingent convertible bonds (CoCos) have become an attractive instrument for banks to seek new capital. The defining characteristic of CoCos is the automatic conversion into common equity when a predetermined trigger is met. Loss absorbing capital is created, which instantly improves the capital structure of distressed banks.\\ 

The thesis scrutinizes the valuation of CoCos. Three major approaches are examined: the structural approach in accordance to \citet{pennacchi2010structural} and both the credit derivatives approach respectively the equity derivatives approach pursuant to \citet{de2011pricing}. The application covers sensitivity analysis to further understand the dynamics of the different methodologies. Based on a case study the viability of those approaches is evaluated.
\end{abstract}