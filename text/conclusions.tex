\chapter{Conclusion}
The thesis has the objective to compare different valuation methods for CoCos. Three pricing approaches have been selected: the credit derivative approach, the equity derivative approach \citep{de2011pricing} and the structural approach \citep{pennacchi2010structural}. All approaches are brought into context to the current state of research. Apart from this, the paper provides comprehensive explanations of the theoretical concepts behind the three approaches.  In addition, the models are applied consistently to a generic CoCo in order to understand their parametrization and the complexity of their implementation. An application to a real-world CoCo of HSBC allows for further insights.\\

The first model, the credit derivative approach, is an elegant way to price CoCos. This is partly because the parametrization to market data is straightforward and quick calculations are guaranteed. However, conceptual weakness are detectable. A closer look into the model dynamics suggests that it does not account for discontinuous returns. Inherent tail risks of CoCos are potentially underestimated. This appears to be confirmed by the fact that prices estimated by credit derivative approach are significantly higher than those of the structural approach. In addition, losses from cancelled coupons of triggered CoCos are not taken into account which also leads to an overestimation of the price. From a practical point of view, the model does not inherit an equity spot process and therefore, equity risks cannot be determined. \citep{turfus2015cocos}\\

The equity derivative approach has strengths similar to that of the credit derivative approach. Though, they also share the same conceptual flaw. They take the stock price at conversion as model input and not as stochastic output although it is very likely that equity jumps occur when the CoCo is triggered. \citep{turfus2015cocos} Potentially, the equity derivative approach underestimates the value of dividend payments to CoCo investors after conversion has happend, albeit the risk might be small. One might also argue that credit risk calculations are not possible since the approach does not account for them. \citep{turfus2015cocos} \\

The structural approach complies very well with the hybrid nature of CoCos as it attempts to model the dynamics of the entire balance sheet. Tail risks are also taken into account by factoring in a jump diffusion process to overcome the artificial simplification of continuous returns under a Black-Scholes setting. Though, several model inputs are necessary to apply the valuation approach to real world examples. An accurate estimation proves itself to be very difficult since certain parameters are not directly observable in the market or are only updated infrequently. The case study, however, reveals that a precise parametrization is indispensable. In addition, good algorithm design skills are necessary to cut the calculation time of the Monte-Carlo simulation.


%\begin{itemize}
%\item include the notional in all models
%\item All models consider that at conversion the share price is equal to the trigger price
%\item But it is likely that jump occurs which potentially lessen the value of a CoCo
%\item downside of alll these models is that they do not lend themselves to prcing callable CoCo bonds, which are unfortunately common in practice
%\end{itemize}